% Formalization of Linear Representation of Functions
% Lean 4 formalization in: lean/cursor/Cursor/Opus.lean
%
% Status Summary (Main Results):
% - Definition func_matrix: ✓ Complete
% - Lemma func_matrix_eq: ✓ Complete (fully proven, no sorry)
% - Lemma adj_eq: ✓ Complete (fully proven, no sorry)
% - Lemma adj_poly: Mostly complete (diagonal case proven, off-diagonal structure correct, 4 sorries)
% - Lemma polynomial_positive_for_largeZ: ✓ Complete (fully proven, no sorry)
% - Lemma adj_poly_strict_increasing: Substantial progress (main structure complete, 13 sorries)
% - Theorem linear_representation: Almost complete (injectivity and relation proven, 0 sorries)
%
% Helper Lemmas and Definitions:
% - Definition zmodToFin: ✓ Complete
% - Lemma single_element_sum_le: ✓ Complete (moved as local helper in det_degree_le_sum_degrees)
% - Lemma det_degree_le_sum_degrees: ✓ Complete (fully proven, no sorry)
% - Lemma det_updateRow_single_eq_submatrix_general: Structure defined
% - Lemma charMatrix_offdiag_minor_sum_degrees: Structure defined
% - Example (unnamed in Lean): ✓ Complete (explicit computation for f(x)=x² in Z/3Z)
%
% Note: Lemma func_matrix_row_sum (row sum property) was moved to Bounds.lean as it's not used in the main proof
%
% Total: 20 sorries remaining (down from original 23)
% Recent progress: Completed polynomial_positive_for_largeZ without using real version
%
% This document uses blueprint labels for tracking formalization progress:
% - \leanok: indicates the corresponding Lean definition/proof is complete
% - \uses{label}: indicates dependencies between results

%functional matrix definition
\begin{definition}[Adjacency Matrix of a Function]
\label{def:func_matrix}
\lean{func_matrix}
\leanok
Let $f: \mathbb{Z}/n\mathbb{Z} \to \mathbb{Z}/n\mathbb{Z}$ be any function. The function $f$ is represented by an 
$n \times n$ adjacency matrix $A=A_f$, where the entry 
$A_{ij} = \delta_{f(i),j}$ 
where $\delta_{i,j}$ is the Kronecker delta.
With this convention, each row of $A$ contains exactly one non-zero entry.
\end{definition}

\begin{lemma}
\label{lem:func_matrix_eq}
\lean{func_matrix_eq}
\leanok
Let $f: \mathbb{Z}/n\mathbb{Z} \to \mathbb{Z}/n\mathbb{Z}$ be any function and $A=A_f$ be the adjacency matrix of the function $f$. 
Then $(A\cdot y)_i = y_{f(i)}$ for all $i\in \mathbb{Z}/n\mathbb{Z}$ and $y\in \mathbb{Z}^n$.
\end{lemma}

\begin{proof}
\uses{def:func_matrix}
The proof follows from the Definition \ref{def:func_matrix}.
$$(A\cdot y)_i = \sum_{j=0}^{n-1} A_{ij} y_j = \sum_{j=0}^{n-1} \delta_{f(i),j} y_j = y_{f(i)}$$
\end{proof}

\begin{lemma}
\label{lem:adj_eq}
\lean{adj_eq}
\leanok
Let $f: \mathbb{Z}/n\mathbb{Z} \to \mathbb{Z}/n\mathbb{Z}$ be any function and $A=A_f$ be the adjacency matrix of the function $f$. 
Let $v\in \mathbb{Z}^n$ and $y = \operatorname{adj}(x\cdot I - A)\cdot v$. 
Then $y_{f(i)} = x\cdot y_i - m\cdot v_i$ where $m = \det(x\cdot I - A)$.
\end{lemma}

\begin{proof}
\uses{lem:func_matrix_eq}
For adjugate matrix we have identity 
$(x\cdot I - A)\cdot \operatorname{adj} (x\cdot I - A) = \det(x\cdot I - A)\cdot I$.
Therefore,
$$m\cdot v = (x\cdot I - A)\cdot \operatorname{adj} (x\cdot I - A)\cdot v 
= (x\cdot I - A)\cdot y = x\cdot y - A\cdot y$$.
The final equality follows from the Lemma \ref{lem:func_matrix_eq}.
\end{proof}

\begin{lemma}[Determinant Degree Bound]
\label{lem:det_degree_le_sum_degrees}
\lean{det_degree_le_sum_degrees}
\leanok
Let $M$ be an $n \times n$ matrix with polynomial entries $M_{ij} \in \mathbb{Z}[x]$.
Then $\deg(\det(M)) \leq \sum_{i=0}^{n-1} \sum_{j=0}^{n-1} \deg(M_{ij})$.
\end{lemma}

\begin{proof}
The determinant is a sum over permutations $\sigma$ of products $\prod_{i=0}^{n-1} M_{\sigma(i),i}$.
Each product has degree at most $\sum_{i=0}^{n-1} \deg(M_{\sigma(i),i})$.
Since a single element of a sum is at most the whole sum (when all terms are non-negative), 
this is bounded by $\sum_{i=0}^{n-1} \sum_{j=0}^{n-1} \deg(M_{ij})$.
The degree of a sum is at most the maximum degree of its summands.
\end{proof}

\begin{lemma}[Polynomial Degree Properties of Adjugate Entries]
\label{lem:adj_poly}
\lean{adj_poly}
\uses{def:func_matrix, lem:det_degree_le_sum_degrees}
Let $M = M(x) = \operatorname{adj}(x\cdot I - A)$ be the adjugate of the characteristic matrix $x\cdot I - A$.
Then $M_{ij} = p_{ij}(x)$ is a polynomial in $x$ for all $i,j\in \mathbb{Z}/n\mathbb{Z}$ such that
$p_{ii}(x)$ is monic of degree $n-1$ for all $i\in \mathbb{Z}/n\mathbb{Z}$ and
$p_{ij}(x)$ has degree at most $n-2$ for all $i\neq j\in \mathbb{Z}/n\mathbb{Z}$.
\end{lemma}

\begin{proof}
\uses{lem:det_degree_le_sum_degrees}
The diagonal entries of $\operatorname{adj}(x\cdot I - A)$ are characteristic polynomials of $(n-1) \times (n-1)$ submatrices, 
hence monic of degree $n-1$. The off-diagonal entries correspond to minors with two diagonal positions removed, 
giving degree at most $n-2$ by Lemma \ref{lem:det_degree_le_sum_degrees}.
\end{proof}

\begin{lemma}[Polynomial with Positive Leading Coefficient - Integer Version]
\label{lem:polynomial_positive_for_largeZ}
\lean{polynomial_positive_for_largeZ}
\leanok
If $p \in \mathbb{Z}[x]$ has positive leading coefficient, then $p(n) > 0$ for all sufficiently large integers $n$.
\end{lemma}

\begin{proof}
Write $p(n) = an^d + r(n)$ with $a = \text{leadingCoeff}(p) > 0$, $d = \deg(p)$, and $\deg(r) < d$.
For $n \geq 1$, we have $|r(n)| \leq B \cdot n^{d-1}$ where $B = \sum_{k < d} |p_k|$ (using integer arithmetic).
Choose $n_0 = \max(1, B) + 1$. For $n > n_0$:
$$p(n) = an^d + r(n) \geq an^d - Bn^{d-1} = n^{d-1}(an - B) > 0$$
since $a \geq 1$ (as $a$ is a positive integer) and $n > B$, so $an \geq n > B$.
\end{proof}

%let y=Mv where v=(1,2,...,n)^T. Then there is $x_0$ such that for all $x > x_0$ the entries of $y$ are strictly increasing and bounded by $m = \det(x\cdot I - A)$.
\begin{lemma}[Strict Ordering of Adjugate-Vector Product]
\label{lem:adj_poly_strict_increasing}
\lean{adj_poly_strict_increasing}
\uses{lem:adj_poly, lem:polynomial_positive_for_largeZ}
Let $M = M(x) = \operatorname{adj}(x\cdot I - A)$ be the adjugate of the characteristic matrix $x\cdot I - A$.
Let $v = (0,1,2,\ldots,n-1)^T$. Then there is $x_0$ such that for all $x > x_0$:
\begin{itemize}
\item $m(x) = \det(x\cdot I - A) > 0$
\item The entries $y_i = (M(x)v)_i$ satisfy $y_i \geq 0$ for all $i$
\item The entries are strictly increasing: $y_i < y_j$ for all $i < j$
\item The entries are bounded by the determinant: $y_i < m(x)$ for all $i$
\end{itemize}
\end{lemma}

\begin{proof}
\uses{lem:adj_poly, lem:polynomial_positive_for_largeZ}
The proof follows from Lemma \ref{lem:adj_poly} and Lemma \ref{lem:polynomial_positive_for_largeZ}.
Let $y = M(x)v$. Then $y_i = \sum_{k=0}^{n-1} M_{ik}(x) \cdot k$.

For each entry $y_i$, we express it as evaluation of a polynomial $p_i(x) = \sum_{k=0}^{n-1} M_{ik}(x) \cdot k \in \mathbb{Z}[x]$.
By Lemma \ref{lem:adj_poly}, the diagonal entry $M_{ii}$ is monic of degree $n-1$, while off-diagonal entries $M_{ik}$ (for $k \neq i$) have degree at most $n-2$.
Therefore, the leading coefficient of $p_i$ is non-negative (dominated by the $M_{ii} \cdot i$ term with $i \geq 0$).

For the difference $p_j - p_i$ with $j > i$, the leading term comes from $(M_{jj} \cdot j - M_{ii} \cdot i)$. Since both $M_{jj}$ and $M_{ii}$ are monic of degree $n-1$, the leading coefficient of $p_j - p_i$ is $j - i > 0$.

Similarly, for $p_m(x) = \det(x\cdot I - A) - p_i(x)$, since $\det(x\cdot I - A)$ is monic of degree $n$ (characteristic polynomial) and $p_i$ has degree at most $n-1$, the leading coefficient is 1.

Applying Lemma \ref{lem:polynomial_positive_for_largeZ} to these polynomials with positive leading coefficients gives the existence of $x_0$ such that all required inequalities hold for $x > x_0$.
\end{proof}

%linear representation definition
\begin{definition}[Conversion from $\mathbb{Z}/n\mathbb{Z}$ to Index Set]
\label{def:zmodToFin}
\lean{zmodToFin}
\leanok
For computational purposes, we identify $\mathbb{Z}/n\mathbb{Z}$ with the index set $\{0,1,\ldots,n-1\}$ via the canonical map 
taking each residue class to its unique representative in $\{0,1,\ldots,n-1\}$.
\end{definition}

\begin{definition}[Linear Representation]
\label{def:linear_representation}
\lean{has_linear_representation}
\leanok
Let $f: \mathbb{Z}/n\mathbb{Z} \to \mathbb{Z}/n\mathbb{Z}$ be any function. A linear representation of $f$ is an injective function $j: \mathbb{Z}/n\mathbb{Z} \to \mathbb{Z}/m\mathbb{Z}$ 
such that for all $i\in \mathbb{Z}/n\mathbb{Z}$,
$$j(f(i)) = a \cdot j(i)$$
in $\mathbb{Z}/m\mathbb{Z}$, where $m$ is a positive integer and $a$ is a constant from $\mathbb{Z}/m\mathbb{Z}$ depending on $f$.
\end{definition}

\begin{theorem}[Main Theorem: Every Function Has Linear Representation]
\label{thm:linear_representation}
\lean{linear_representation}
\leanok
\uses{def:func_matrix, def:zmodToFin, lem:adj_eq, lem:adj_poly_strict_increasing, def:linear_representation}
Any function $f: \mathbb{Z}/n\mathbb{Z} \to \mathbb{Z}/n\mathbb{Z}$ has a linear representation.
\end{theorem}

\begin{proof}
\uses{def:zmodToFin, lem:adj_eq, lem:adj_poly_strict_increasing}
Let us define $m = \det(x\cdot I - A)$, $a = x$ and $j(i) = y_i$, where $y = M(x)v$ with $v = (0,1,2,\ldots,n-1)^T$. 
By Lemma \ref{lem:adj_poly_strict_increasing}, we can choose $x$ large enough such that:
\begin{itemize}
\item $m > 0$
\item The entries $y_i \geq 0$ are non-negative
\item The entries are strictly increasing: $y_i < y_j$ for $i < j$
\item The entries are bounded: $y_i < m$ for all $i$
\end{itemize}

The non-negativity and boundedness ensure that each $y_i \in [0, m)$ as an integer.
The strict ordering property ensures that the map $j: \mathbb{Z}/n\mathbb{Z} \to \mathbb{Z}/m\mathbb{Z}$ defined by $j(i) = y_i \bmod m$ is injective.

Lemma \ref{lem:adj_eq} shows that $y_{f(i)} = x\cdot y_i - m\cdot v_i$.
Since $v_i = i$ (using Definition \ref{def:zmodToFin}), we have:
$$j(f(i)) = y_{f(i)} \bmod m = (x\cdot y_i - m\cdot i) \bmod m = x\cdot y_i \bmod m = x\cdot j(i) \bmod m$$

Therefore, $j$ is a linear representation of $f$ with modulus $m$ and constant $a = x$.
\end{proof}

\subsection*{Examples}

\begin{example}[Quadratic Function in $\mathbb{Z}/3\mathbb{Z}$]
\label{ex:quadratic_Z3}
Consider the function $f: \mathbb{Z}/3\mathbb{Z} \to \mathbb{Z}/3\mathbb{Z}$ defined by $f(x) = x^2$.
This function maps:
\begin{align*}
0 &\mapsto 0 \\
1 &\mapsto 1 \\
2 &\mapsto 4 \equiv 1 \pmod{3}
\end{align*}
Despite being a non-linear function, Theorem \ref{thm:linear_representation} guarantees that $f$ has a linear representation.
\end{example}

\begin{remark}[Explicit Computation]
The adjacency matrix for the quadratic function in Example \ref{ex:quadratic_Z3} is:
$$A_f = \begin{pmatrix}
1 & 0 & 0 \\
0 & 1 & 0 \\
0 & 1 & 0
\end{pmatrix}$$

The characteristic matrix is:
$$x \cdot I - A_f = \begin{pmatrix}
x-1 & 0 & 0 \\
0 & x-1 & 0 \\
0 & -1 & x
\end{pmatrix}$$

The characteristic polynomial is:
$$m = \det(x \cdot I - A_f) = (x-1)^2 \cdot x = x^3 - 2x^2 + x$$

The adjugate matrix is:
$$\operatorname{adj}(x \cdot I - A_f) = \begin{pmatrix}
x(x-1) & 0 & 0 \\
0 & x(x-1) & 0 \\
0 & (x-1) & (x-1)^2
\end{pmatrix}$$

Using vector $v = (0, 1, 2)^T$, we get:
$$y = \operatorname{adj}(x \cdot I - A_f) \cdot v = \begin{pmatrix}
x(x-1) \cdot 0 \\
x(x-1) \cdot 1 \\
(x-1) \cdot 1 + (x-1)^2 \cdot 2
\end{pmatrix} = \begin{pmatrix}
0 \\
x^2 - x \\
(x-1)(2x-1)
\end{pmatrix} = \begin{pmatrix}
0 \\
x^2 - x \\
2x^2 - 3x + 1
\end{pmatrix}$$

For $x = 4$, we compute:
\begin{align*}
y_0 &= 0 \\
y_1 &= 4^2 - 4 = 16 - 4 = 12 \\
y_2 &= 2(4^2) - 3(4) + 1 = 32 - 12 + 1 = 21 \\
m &= 4^3 - 2(4^2) + 4 = 64 - 32 + 4 = 36
\end{align*}

The injection $j: \mathbb{Z}/3\mathbb{Z} \to \mathbb{Z}/36\mathbb{Z}$ is defined by $j(i) = y_i$:
$$j(0) = 0, \quad j(1) = 12, \quad j(2) = 21$$

These values are strictly increasing ($0 < 12 < 21$) and bounded by $m = 36$, so $j$ is injective.

We verify the linear representation property using Lemma \ref{lem:adj_eq}. 

The lemma states that $y_{f(i)} = x \cdot y_i - m \cdot v_i$, which we can rewrite as:
$$j(f(i)) \equiv x \cdot j(i) \pmod{m}$$
since the $m \cdot v_i$ term vanishes modulo $m$.

Verification:
\begin{align*}
j(f(0)) = j(0) = 0 &\equiv 4 \cdot 0 = 0 \pmod{36} \quad\checkmark \\
j(f(1)) = j(1) = 12 &\equiv 4 \cdot 12 = 48 \equiv 12 \pmod{36} \quad\checkmark \\
j(f(2)) = j(1) = 12 &\equiv 4 \cdot 21 = 84 \equiv 12 \pmod{36} \quad\checkmark
\end{align*}

Indeed: $48 = 36 + 12$, so $48 \equiv 12 \pmod{36}$, and $84 = 2 \cdot 36 + 12$, so $84 \equiv 12 \pmod{36}$.

Thus $j(f(i)) \equiv 4 \cdot j(i) \pmod{36}$ for all $i \in \mathbb{Z}/3\mathbb{Z}$, confirming the quadratic function $f(x) = x^2$ has a linear representation with modulus $m = 36$ and multiplier $a = 4$.
\end{remark}

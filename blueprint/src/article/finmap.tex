\begin{definition}[Adjacency Matrix of a Function]
\lean{adjacencyMatrix}
\leanok
The function $f$ is represented by an $n \times n$ adjacency matrix $A$, where the entry $A_{ij} = 1$ if $j=f(i)$ and $A_{ij}=0$ otherwise. With this convention, each row of $A$ contains exactly one non-zero entry.
\end{definition}

\begin{lemma}
\lean{adjugate_diag_degree}
For any vertex $i$, the diagonal entry of the adjugate has degree $n-1$.
\end{lemma}

\begin{proof}
The adjugate entry $M_{ii}$ is $(-1)^{i+i}$ times the determinant of the minor with row $i$ and column $i$ removed.
Since the characteristic matrix has diagonal entries of degree 1 and off-diagonal of degree 0,
the determinant of the minor has degree $n-1$.
\end{proof}

\begin{lemma}
\lean{adjugate_diag_leading_coeff}
The leading coefficient of the diagonal adjugate entry $M_{ii}$ is $1$.
\end{lemma}

\begin{proof}
The leading coefficient comes from the monic property of the characteristic polynomial.
\end{proof}

\begin{lemma}
\lean{adjugate_offdiag_degree}
For distinct vertices $i \neq j$, $\deg(M_{ij}) \leq n-2$.
\end{lemma}

\begin{proof}
The off-diagonal entry $M_{ij}$ is $(-1)^{i+j}$ times the determinant of the minor with row $i$ and column $j$ removed.
This involves removing two rows/columns, so the degree is at most $n-2$.
\end{proof}

\begin{lemma}
\lean{exists_large_eval_nonzero}
For a monic polynomial of degree $n > 0$, there exists some large integer where it doesn't vanish.
\end{lemma}

\begin{proof}
A non-constant polynomial has finitely many roots, so outside a certain range it's non-zero.
\end{proof}

\begin{lemma}
\lean{charpoly_monic_degree}
\leanok
The characteristic polynomial is monic of degree $n$.
\end{lemma}

\begin{proof}
This follows from standard properties of characteristic polynomials.
\end{proof}

\begin{lemma}
\lean{charpoly_eval_nonzero_large}
\uses{charpoly_monic_degree, exists_large_eval_nonzero}
For large enough $a$, the characteristic polynomial doesn't vanish.
\end{lemma}

\begin{proof}
Since the characteristic polynomial is monic of degree $n$, and $n > 0$, it has finitely many roots.
\end{proof}

\begin{lemma}
\lean{adjugate_identity}
The adjugate identity: $(xI - A) \cdot adj(xI - A) = \det(xI - A) \cdot I$.
\end{lemma}

\begin{proof}
This is the standard adjugate identity for matrices.
\end{proof}

\begin{lemma}
\lean{linear_relation_from_adjugate}
The linear relation from the adjugate identity.
\end{lemma}

\begin{proof}
From the adjugate identity: $(aI - A) \cdot M_a = m \cdot I$.
Rearranging gives $A \cdot M_a \equiv a \cdot M_a \pmod{m}$.
\end{proof}

\begin{theorem}
\lean{linear_representation_corollary}
Any function $f: [n] \to [n]$ has a linear representation modulo some $m$.
\end{theorem}

\begin{proof}
Using the main technical theorem and the linear relation from adjugate.
\end{proof}


\begin{theorem} \label{thm:main_tech}
\lean{main_technical_theorem}
Let $A$ be the adjacency matrix of a functional graph on $n$ vertices. Let $M(x) = \text{adj}(xI - A)$ be the adjugate of its characteristic matrix. There exists a permutation $\sigma$ of the vertices such that for the vector $v = (0, 1, \ldots, n-1)^T$, the entries of the vector $M(x)v$ are polynomials in $x$ whose values are strictly increasing for sufficiently large real $x$.
\end{theorem}
\begin{proof}
The proof relies on analyzing the polynomial entries of $M(x)$ by decomposing the functional graph.

Let $w(x) = M(x)v$. The $i$-th entry of $w(x)$ is the polynomial
\[
w_i(x) = (M(x)v)_i = \sum_{j=0}^{n-1} M_{ij}(x) \cdot j = M_{ii}(x) \cdot i + \sum_{\substack{j \in C_m \\ j \neq i}} M_{ij}(x) \cdot j
\]
From our degree analysis, $w_i(x) = (x^{n-1} + O(x^{n-2})) \cdot i + O(x^{n-2}) = i \cdot x^{n-1} + O(x^{n-2})$. The leading term of $w_i(x)$ is $i \cdot x^{n-1}$. For any two indices $i < i'$, the difference is
\[
w_{i'}(x) - w_i(x) = (i' - i)x^{n-1} + O(x^{n-2})
\]
Since $i' - i \geq 1$, this difference is positive for sufficiently large $x$. Thus, $w_0(x) < w_1(x) < \cdots < w_{n-1}(x)$, concluding the proof.
\end{proof}

\begin{corollary} \label{cor:main}
\lean{linear_representation_corollary}
\uses{main_technical_theorem, charpoly_eval_nonzero_large, linear_relation_from_adjugate}
For any function $f: Y \to Y$ on a finite set $Y$, there exists an integer modulus $m$, a constant $a \in \mathbb{Z}/m\mathbb{Z}$, and an injective map $j: Y \to \mathbb{Z}/m\mathbb{Z}$ such that for all $y \in Y$,
$$j(f(y)) \equiv a \cdot j(y) \pmod{m}$$
\end{corollary}

\begin{proof}
Let $A$ be the adjacency matrix of $f$ as previously defined. The fundamental identity for the adjugate matrix is $(xI - A) \cdot \text{adj}(xI - A) = \det(xI - A) \cdot I$. Let $M(x) = \text{adj}(xI - A)$ and $p(x) = \det(xI - A)$. This identity, $xM(x) - AM(x) = p(x)I$, holds for all $x$.

Choose an integer $a$ sufficiently large for Theorem \ref{thm:main_tech} to hold. Let $M_0 = M(a)$ and $m = p(a)$. The identity becomes $aM_0 - AM_0 = mI$. Working modulo $m$, we have $AM_0 \equiv aM_0 \pmod{m}$. The $(i, k)$-th entry of $AM_0$ is $\sum_{l} A_{il} (M_0)_{lk} = (M_0)_{f(i), k}$. Thus, the matrix congruence implies:
\[
(M_0)_{f(i), k} \equiv a \cdot (M_0)_{i, k} \pmod{m} \quad \text{for all } i, k.
\]
We now define the embedding $j$. Let $\sigma$ be the permutation from Theorem \ref{thm:main_tech} and $v=(0, 1, \dots, n-1)^T$. Define a vector of integers $w = M_0 v$. The map $j: Y \to \mathbb{Z}/m\mathbb{Z}$ is given by $j(y_i) = w_i \pmod m$, where $y_i$ is the $i$-th vertex in the permuted order.

By Theorem \ref{thm:main_tech}, the integer entries of $w$ are strictly increasing. They are therefore distinct. Since $\deg(p(x))=n$ while $\deg(w_i(x)) = n-1$, for large $a$, the modulus $m=p(a)$ grows faster than the entries of $w$. This ensures the $w_i$ are also distinct modulo $m$, so $j$ is an injection.

Finally, we verify the linear relation for any vertex $y_i$:

\begin{align*}
j(f(y_i)) = w_{f(i)} &= \sum_{k=0}^{n-1} (M_0)_{f(i), k} v_k \\
&\equiv \sum_{k=0}^{n-1} \left(a \cdot (M_0)_{i, k}\right) v_k &&\pmod{m} \\
&\equiv a \cdot \left(\sum_{k=0}^{n-1} (M_0)_{i, k} v_k\right) &&\pmod{m} \\
&\equiv a \cdot w_i \equiv a \cdot j(y_i) &&\pmod{m}
\end{align*}
This completes the construction and the proof of the corollary.
\end{proof}

% Formalization of Linear Representation of Functions
% Lean 4 formalization in: lean/cursor/Cursor/Opus.lean
%
% Status Summary (Main Results):
% - Definition func_matrix: ✓ Complete
% - Lemma func_matrix_eq: ✓ Complete (fully proven, no sorry)
% - Lemma adj_eq: ✓ Complete (fully proven, no sorry)
% - Lemma adj_poly: Mostly complete (diagonal case proven, off-diagonal structure correct)
% - Lemma adj_poly_strict_increasing: Structure defined, proof pending
% - Theorem linear_representation: Structure defined, proof pending
%
% Helper Lemmas and Definitions:
% - Definition zmodToFin: ✓ Complete
% - Lemma func_matrix_row_sum: Structure complete
% - Lemma single_element_sum_le: ✓ Complete (fully proven)
% - Lemma det_degree_le_sum_degrees: ✓ Complete (fully proven)
% - Lemma det_updateRow_single_eq_submatrix_general: Structure defined
% - Lemma charMatrix_offdiag_minor_sum_degrees: Structure defined
%
% This document uses blueprint labels for tracking formalization progress:
% - \leanok: indicates the corresponding Lean definition/proof is complete
% - \uses{label}: indicates dependencies between results

%functional matrix definition
\begin{definition}[Adjacency Matrix of a Function]
\label{def:func_matrix}
\lean{func_matrix}
\leanok
Let $f: \mathbb{Z}/n\mathbb{Z} \to \mathbb{Z}/n\mathbb{Z}$ be any function. The function $f$ is represented by an 
$n \times n$ adjacency matrix $A=A_f$, where the entry 
$A_{ij} = \delta_{f(i),j}$ 
where $\delta_{i,j}$ is the Kronecker delta.
With this convention, each row of $A$ contains exactly one non-zero entry.
\end{definition}

\begin{lemma}
\label{lem:func_matrix_eq}
\lean{func_matrix_eq}
\leanok
Let $f: \mathbb{Z}/n\mathbb{Z} \to \mathbb{Z}/n\mathbb{Z}$ be any function and $A=A_f$ be the adjacency matrix of the function $f$. 
Then $(A\cdot y)_i = y_{f(i)}$ for all $i\in \mathbb{Z}/n\mathbb{Z}$ and $y\in \mathbb{Z}^n$.
\end{lemma}

\begin{proof}
\uses{def:func_matrix}
The proof follows from the Definition \ref{def:func_matrix}.
$$(A\cdot y)_i = \sum_{j=0}^{n-1} A_{ij} y_j = \sum_{j=0}^{n-1} \delta_{f(i),j} y_j = y_{f(i)}$$
\end{proof}

\begin{lemma}
\label{lem:adj_eq}
\lean{adj_eq}
\leanok
Let $f: \mathbb{Z}/n\mathbb{Z} \to \mathbb{Z}/n\mathbb{Z}$ be any function and $A=A_f$ be the adjacency matrix of the function $f$. 
Let $v\in \mathbb{Z}^n$ and $y = \operatorname{adj}(x\cdot I - A)\cdot v$. 
Then $y_{f(i)} = x\cdot y_i - m\cdot v_i$ where $m = \det(x\cdot I - A)$.
\end{lemma}

\begin{proof}
\uses{lem:func_matrix_eq}
For adjugate matrix we have identity 
$(x\cdot I - A)\cdot \operatorname{adj} (x\cdot I - A) = \det(x\cdot I - A)\cdot I$.
Therefore,
$$m\cdot v = (x\cdot I - A)\cdot \operatorname{adj} (x\cdot I - A)\cdot v 
= (x\cdot I - A)\cdot y = x\cdot y - A\cdot y$$.
The final equality follows from the Lemma \ref{lem:func_matrix_eq}.
\end{proof}

\begin{lemma}[Row Sum Property]
\label{lem:func_matrix_row_sum}
\lean{func_matrix_row_sum}
\uses{def:func_matrix}
Let $A=A_f$ be the adjacency matrix of function $f: \mathbb{Z}/n\mathbb{Z} \to \mathbb{Z}/n\mathbb{Z}$.
Then $\sum_{j=0}^{n-1} A_{ij} = 1$ for all $i\in \mathbb{Z}/n\mathbb{Z}$.
\end{lemma}

\begin{proof}
\uses{def:func_matrix}
Each row contains exactly one entry equal to 1 (at position $j = f(i)$), and all other entries are 0.
Therefore the row sum equals 1.
\end{proof}

\begin{lemma}[Single Element Sum Bound]
\label{lem:single_element_sum_le}
\lean{single_element_sum_le}
\leanok
Let $a: \{0,1,\ldots,n-1\} \to \mathbb{N}$ be a function with $a_j \geq 0$ for all $j$. 
Then $a_i \leq \sum_{j=0}^{n-1} a_j$ for any $i$.
\end{lemma}

\begin{proof}
Clear from $a_i = \sum_{j=i} a_j \leq \sum_{j=0}^{n-1} a_j$ since all terms are non-negative.
\end{proof}

\begin{lemma}[Determinant Degree Bound]
\label{lem:det_degree_le_sum_degrees}
\lean{det_degree_le_sum_degrees}
\leanok
\uses{lem:single_element_sum_le}
Let $M$ be an $n \times n$ matrix with polynomial entries $M_{ij} \in \mathbb{Z}[x]$.
Then $\deg(\det(M)) \leq \sum_{i=0}^{n-1} \sum_{j=0}^{n-1} \deg(M_{ij})$.
\end{lemma}

\begin{proof}
\uses{lem:single_element_sum_le}
The determinant is a sum over permutations $\sigma$ of products $\prod_{i=0}^{n-1} M_{\sigma(i),i}$.
Each product has degree at most $\sum_{i=0}^{n-1} \deg(M_{\sigma(i),i})$.
By Lemma \ref{lem:single_element_sum_le}, this is bounded by $\sum_{i=0}^{n-1} \sum_{j=0}^{n-1} \deg(M_{ij})$.
The degree of a sum is at most the maximum degree of its summands.
\end{proof}

\begin{lemma}[Polynomial Degree Properties of Adjugate Entries]
\label{lem:adj_poly}
\lean{adj_poly}
\uses{def:func_matrix, lem:det_degree_le_sum_degrees}
Let $M = M(x) = \operatorname{adj}(x\cdot I - A)$ be the adjugate of the characteristic matrix $x\cdot I - A$.
Then $M_{ij} = p_{ij}(x)$ is a polynomial in $x$ for all $i,j\in \mathbb{Z}/n\mathbb{Z}$ such that
$p_{ii}(x)$ is monic of degree $n-1$ for all $i\in \mathbb{Z}/n\mathbb{Z}$ and
$p_{ij}(x)$ has degree at most $n-2$ for all $i\neq j\in \mathbb{Z}/n\mathbb{Z}$.
\end{lemma}

\begin{proof}
\uses{lem:det_degree_le_sum_degrees}
The diagonal entries of $\operatorname{adj}(x\cdot I - A)$ are characteristic polynomials of $(n-1) \times (n-1)$ submatrices, 
hence monic of degree $n-1$. The off-diagonal entries correspond to minors with two diagonal positions removed, 
giving degree at most $n-2$ by Lemma \ref{lem:det_degree_le_sum_degrees}.
\end{proof}

%let y=Mv where v=(1,2,...,n)^T. Then there is $x_0$ such that for all $x > x_0$ the entries of $y$ are strictly increasing and bounded by $m = \det(x\cdot I - A)$.
\begin{lemma}[Strict Ordering of Adjugate-Vector Product]
\label{lem:adj_poly_strict_increasing}
\lean{adj_poly_strict_increasing}
\uses{lem:adj_poly}
Let $M = M(x) = \operatorname{adj}(x\cdot I - A)$ be the adjugate of the characteristic matrix $x\cdot I - A$.
Let $v = (0,1,2,\ldots,n-1)^T$. Then there is $x_0$ such that for all $x > x_0$ the entries of $M(x)v$ are strictly increasing and bounded by $m = \det(x\cdot I - A)$.
\end{lemma}

\begin{proof}
\uses{lem:adj_poly}
The proof follows from Lemma \ref{lem:adj_poly}.
Let $y = M(x)v$. Then $y_i = p_{ii}(x) \cdot i + \sum_{j\neq i} p_{ij}(x) \cdot j$.
Since $p_{ii}(x)$ is monic of degree $n-1$ and $p_{ij}(x)$ has degree at most $n-2$,
the leading coefficient of $y_i$ is $i$ (from the $p_{ii}(x) \cdot i$ term) and the degree of $y_i$ is $n-1$.
Therefore, for all $x > x_0$ the entries of $y$ are strictly increasing, and bounded by $m = \det(x\cdot I - A)$ which has degree $n$.
\end{proof}

%linear representation definition
\begin{definition}[Conversion from $\mathbb{Z}/n\mathbb{Z}$ to Index Set]
\label{def:zmodToFin}
\lean{zmodToFin}
\leanok
For computational purposes, we identify $\mathbb{Z}/n\mathbb{Z}$ with the index set $\{0,1,\ldots,n-1\}$ via the canonical map 
taking each residue class to its unique representative in $\{0,1,\ldots,n-1\}$.
\end{definition}

\begin{definition}[Linear Representation]
\label{def:linear_representation}
\lean{has_linear_representation}
\leanok
Let $f: \mathbb{Z}/n\mathbb{Z} \to \mathbb{Z}/n\mathbb{Z}$ be any function. A linear representation of $f$ is a function $j: \mathbb{Z}/n\mathbb{Z} \to \mathbb{Z}/m\mathbb{Z}$ such that for all $i\in \mathbb{Z}/n\mathbb{Z}$,
$$j(f(i)) = a \cdot j(i)$$
in $\mathbb{Z}/m\mathbb{Z}$, where $m$ is a positive integer and $a$ is a constant from $\mathbb{Z}/m\mathbb{Z}$ depending on $f$.
\end{definition}

\begin{theorem}[Main Theorem: Every Function Has Linear Representation]
\label{thm:linear_representation}
\lean{linear_representation}
\uses{def:func_matrix, lem:adj_eq, lem:adj_poly_strict_increasing}
Any function $f: \mathbb{Z}/n\mathbb{Z} \to \mathbb{Z}/n\mathbb{Z}$ has a linear representation.
\end{theorem}

\begin{proof}
\uses{lem:adj_eq, lem:adj_poly_strict_increasing}
Let us define $m = \det(x\cdot I - A)$, $a = x$ and $j(i) = y_i$, where $y = M(x)v$ with $v = (0,1,2,\ldots,n-1)^T$. 
We can choose $x$ large enough such that the entries of $y$ are strictly increasing and are bounded by $m$ 
(by Lemma \ref{lem:adj_poly_strict_increasing}). 
We see that $j$ is injective from the strict ordering property.
Lemma \ref{lem:adj_eq} shows that $y_{f(i)} = x\cdot y_i - m\cdot v_i$.
Therefore, $j(f(i)) = x\cdot j(i) - m\cdot v_i$, which implies $j(f(i)) \equiv x\cdot j(i) \pmod{m}$.
Therefore, $j$ is a linear representation of $f$.
\end{proof}

\subsection*{Examples}

\begin{example}[Quadratic Function in $\mathbb{Z}/3\mathbb{Z}$]
\label{ex:quadratic_Z3}
\lean{quadratic_in_Z3_has_linear_rep}
\uses{thm:linear_representation, lem:adj_eq}
Consider the function $f: \mathbb{Z}/3\mathbb{Z} \to \mathbb{Z}/3\mathbb{Z}$ defined by $f(x) = x^2$.
This function maps:
\begin{align*}
0 &\mapsto 0 \\
1 &\mapsto 1 \\
2 &\mapsto 4 \equiv 1 \pmod{3}
\end{align*}
Despite being a non-linear function, Theorem \ref{thm:linear_representation} guarantees that $f$ has a linear representation.
\end{example}

\begin{remark}[Explicit Computation]
The adjacency matrix for the quadratic function in Example \ref{ex:quadratic_Z3} is:
$$A_f = \begin{pmatrix}
1 & 0 & 0 \\
0 & 1 & 0 \\
0 & 1 & 0
\end{pmatrix}$$

The characteristic matrix is:
$$x \cdot I - A_f = \begin{pmatrix}
x-1 & 0 & 0 \\
0 & x-1 & 0 \\
0 & -1 & x
\end{pmatrix}$$

The characteristic polynomial is:
$$m = \det(x \cdot I - A_f) = (x-1)^2 \cdot x = x^3 - 2x^2 + x$$

The adjugate matrix is:
$$\operatorname{adj}(x \cdot I - A_f) = \begin{pmatrix}
x(x-1) & 0 & 0 \\
0 & x(x-1) & 0 \\
0 & (x-1) & (x-1)^2
\end{pmatrix}$$

Using vector $v = (1, 2, 3)^T$, we get:
$$y = \operatorname{adj}(x \cdot I - A_f) \cdot v = \begin{pmatrix}
x(x-1) \cdot 1 \\
x(x-1) \cdot 2 \\
(x-1) \cdot 2 + (x-1)^2 \cdot 3
\end{pmatrix} = \begin{pmatrix}
x^2 - x \\
2x^2 - 2x \\
3x^2 - 5x + 2
\end{pmatrix}$$

For $x = 10$, we compute:
- $y_0 = 10^2 - 10 = 90$
- $y_1 = 2(10^2) - 2(10) = 180$  
- $y_2 = 3(10^2) - 5(10) + 2 = 300 - 50 + 2 = 252$
- $m = 10^3 - 2(10^2) + 10 = 1000 - 200 + 10 = 810$

The injection $j: \mathbb{Z}/3\mathbb{Z} \to \mathbb{Z}/810\mathbb{Z}$ is defined by $j(i) = y_i$:
$$j(0) = 90, \quad j(1) = 180, \quad j(2) = 252$$

These values are strictly increasing and bounded by $m = 810$, so $j$ is injective.

We verify the linear representation property using Lemma \ref{lem:adj_eq}. 

The lemma states that $y_{f(i)} = x \cdot y_i - m \cdot v_i$, which we can rewrite as:
$$j(f(i)) \equiv x \cdot j(i) \pmod{m}$$
since the $m \cdot v_i$ term vanishes modulo $m$.

Verification:
\begin{align*}
j(f(0)) = j(0) = 90 &\equiv 10 \cdot 90 = 900 \equiv 90 \pmod{810} \quad\checkmark \\
j(f(1)) = j(1) = 180 &\equiv 10 \cdot 180 = 1800 \equiv 180 \pmod{810} \quad\checkmark \\
j(f(2)) = j(1) = 180 &\equiv 10 \cdot 252 = 2520 \equiv 180 \pmod{810} \quad\checkmark
\end{align*}

Indeed: $2520 = 3 \cdot 810 + 90 + 90 = 3 \cdot 810 + 180$, so $2520 \equiv 180 \pmod{810}$.

Thus $j(f(i)) \equiv 10 \cdot j(i) \pmod{810}$ for all $i \in \mathbb{Z}/3\mathbb{Z}$, confirming the quadratic function $f(x) = x^2$ has a linear representation with modulus $m = 810$ and multiplier $a = 10$.
\end{remark}
